\writer{Rafaela, Ruxandra}
The first step when starting to work on the project was to decide upon the general architecture of our
implementation. When dealing with a system of this amplitude it is best to start testing individual components
as soon as possible. Module testing ensures that everything that needs to be integrated, has been previously
tested and verified.

Besides ensuring the modularity of the system, a decision that has to be made from the beginning is whether to
use \textbf{multibody planning} or \textbf{multi-agent planning}. While both have their advantages, each of
them is applicable in different scenarios. Multibody planning is preferred when dealing with an environment
where state exploration is very efficient due to low branching factor. However, when planning on a large scale
with many possible alternative actions, where the branching factor is very large, a solution based on
multi-agent systems reacts better.

The next step was analysing to which extent the environment, the number of agents and the unpredictable
obstacles affect the performance and the decision making process. Before starting to compute the plan for
achieving the goals, a \textbf{pre-analysis} of the existent paths, patterns and goal, as well as the
complexity of the environment must be made.

Even though a good knowledge of the map is useful, the way the states are explored can dramatically affect the
performance. When doing blind state exploration the branching factor can exponentially escalate. To create a
good heuristic, the data from the pre-analysis should be used, though a comprehensive analysis of the  current
state is essential. However, considering the high branching factor, a complete analysis of the current state
would take an unacceptable amount of time. Aside from time, memory resources need to be kept in mind when
dealing with high branching factors, as very large number of states are being generated. Thus, a
\textbf{compromise} between having the optimal solution and finding the least computationally expensive
solution needs to be found. Compromises include using \textbf{heuristics} and \textbf{pruning} of states
during the exploration.

The problems analysed up to this point regard mostly single agent systems. In the case of multi-agent systems,
such as the one we are dealing with, other problems should be considered. Even when dealing with only two
agents, cases where the agents have conflicting plans might occur. In some cases, the \textbf{conflict}
appears just because the agents have not chosen their goals in the proper order. This can easily be solved by
\textbf{reordering/ replanning} of goals. In other cases, in order to avoid conflicts, the plans of the agents
could be merged by \textbf{postponing actions}. This is not always possible as two agents might need to access
common resources. Furthermore, for very big plans or for a large number of agents, the number of possibilities
to interlace the agents’ plans is very big and, as such, it could lead to wasting a lot of time and resources
on computing the solution. When plan merging is not an option, other approaches can be taken. Among them,
there are the \textbf{inter-agent communication} and the \textbf{high-level conflict solving}.

The communication between agents can also prove to be problematic as it can introduce synchronization problems
which can lead to cases like multiple agents being assigned the same goal or a goal not being assigned to any
agent. For this problem, solutions that involve direct interagent communication like
FIPA-ACL\footnote{FIPA-ACL \url{http://www.fipa.org/repository/aclspecs.html}}  can be used, as well as
solutions that involve communication being performed through a central entity.

When conflicts arise, there is the option to make agents communicate in order to solve the conflict and decide
upon a plan or the option for the agents to resort to a central entity that tells them what to do in order to
solve the conflict. In some cases, a hybrid solution between the two can also be used.

For the case where unknown objects are present in the environment, special approaches need to be taken into
consideration. As replanning could be triggered at any moment due to detection of unknown objects, it is not
that important to have a complete and optimal plan, but to be able to compute an acceptable plan in a short
amount of time. Also, when dealing with multiple agents, not all agents need to recompute the plan when
another agent identifies an unknown object.

As it is expected, the number of things that need to be taken into consideration for elaborating efficient and
optimal solutions is very large and a compromise should always be found between all the factors involved, as
some will prove to be more important in some cases while others in other cases.


